%
%                  Politecnico di Milano
%
%         Student: Caravano Andrea, Alberto Cantele
%            A.Y.: 2024/2025
%
%   Last modified: 25/05/2025
%
%     Description: Internet of Things: Homework
%                  Exercise n. 2
%

\documentclass[a4paper,11pt]{article} % tipo di documento
\usepackage[T1]{fontenc} % codifica dei font
\usepackage[utf8]{inputenc} % lettere accentate da tastiera
\usepackage[english]{babel} % lingua del documento
\usepackage{lipsum} % genera testo fittizio
\usepackage{url} % per scrivere gli indirizzi Internet e/o di riferimento nella pagina

\usepackage{hyperref} % per modificare il comportamento dei collegamenti ipertestuali

\usepackage[margin=0.7in]{geometry} % margine di pagina

\usepackage{graphicx} % per inserire immagini

\usepackage{minted} % per colorazione automatica del codice (installare pygments da Homebrew)
% \usepackage{pythonhighlight} % per Python

\setminted{ % si può impostare il linguaggio specifico con \setminted[JSON] ad esempio
    linenos=true,
    breaklines=true,
    encoding=utf8,
    fontsize=\normalsize,
    frame=lines
}

\usepackage{fancyhdr}
\usepackage{textcomp}
\usepackage{siunitx} % per gestione intestazione e piè di pagina

\usepackage{tcolorbox} % per riquadrature di vario colore

\usepackage{float} % per figure comparative flottanti
\usepackage{amsmath} % per frazioni in display style

\usepackage{icomma} % virgola come separatore decimale

\usepackage{titlesec} % per configurazione del tipo paragrafo

\usepackage{multirow} % righe multiple in tabelle

\usepackage{subfig} % descrizione sottostante a figure

% setup del tipo paragrafo
\setcounter{secnumdepth}{4}

\titleformat{\paragraph}
{\normalfont\normalsize\bfseries}{\theparagraph}{1em}{}
\titlespacing*{\paragraph}
{0pt}{3.25ex plus 1ex minus .2ex}{1.5ex plus .2ex}

\hypersetup{ % metadati di titolo e autore nel PDF
    hidelinks, % leva colore attorno collegamenti ipertestuali
    pdftitle={Internet of Things: Homework - exercise n. 2},
    pdfauthor={Andrea Caravano, Alberto Cantele}
}

\setlength{\parindent}{0pt} % rimuove l'indentazione del testo

\tcbset{ % impostazioni per riquadrature
    colback=gray!20,
    colframe=black,
    boxrule=0.5pt
}

\captionsetup{labelformat=empty} % rimuove la caption delle figure

% Imposta la profondità dell'indice a 2 livelli (sottosezioni, non sotto-sottosezioni)
\setcounter{tocdepth}{2}

% Definizione comandi per floor e ceiling
\newcommand{\floor}[1]{\left\lfloor #1 \right\rfloor}
\newcommand{\ceil}[1]{\left\lceil #1 \right\rceil}

\pagestyle{fancy}
\fancyhead{}\fancyfoot{}
\fancyhead[L]{\textbf{Internet of Things: Homework - exercise n. 2}}
\fancyhead[R]{Andrea Caravano, Alberto Cantele}
\fancyfoot[C]{\thepage}

\title{\textbf{Internet of Things}\\Homework: exercise n. 2}
\author{Andrea Caravano, Alberto Cantele}
\date{Academic Year 2024--25}

\begin{document}
\maketitle

%\tableofcontents

\section*{Exercise text}\label{exercise-text}

Consider the following pseudocode for a ESP32-based IoT monitoring system

\begin{minted}{Python}
// Global Timer Handle
declare timer_handle as esp_timer_handle_t

// Initialization
function setup_camera():
    initialize_camera(QVGA)

function setup_timer():
    declare timer_config as esp_timer_create_args_t
    set timer_config.callback to process_frame
    set timer_config.name to "10_sec_timer"
    call esp_timer_create(&timer_config, &timer_handle)
    call esp_timer_start_periodic(timer_handle, 10_000_000) // 10s

function app_main():
    call setup_camera()
    call setup_timer()
    loop forever:
        delay(100 ms)

// Called every 10 seconds
function process_frame(arg):
    image = capture_camera_frame()
    person_count = estimate_number_of_people(image)
    if person_count == 0:
        payload = create_message(size=1KB)
    else if person_count == 1:
        payload = create_message(size=3KB)
    else:
        payload = create_message(size=6KB)
\end{minted}

Assuming the system is operated with IEEE 802.15.4 in beacon-enabled mode (CFP only) and that the number of people present in the camera frame at any instant follows a Poisson distribution with an average rate of $\lambda = 0.15$ people/frame

\begin{enumerate}
    \item Compute the Probability Mass Function of the output rate of the ESP32 P($r = r_0$), P($r = r_1$), P($r = r_2$), where $r_0$, $r_1$ and $r_2$ are the output rates when there are 0, 1 or more than 1 people in the captured frame, respectively.
    \item Based on the output rate PMF, compute a consistent slot assignment for the CFP in a monitoring system composed of 1 PAN coordinator and 3 camera nodes. Assume nominal bit rate $R=250$ kbps, packets of $L=128$ bytes, 1 packet fits exactly in one slot. Compute $T_S$ (slot time), Number of slots in the CFP, $T_{active}$, $T_{inactive}$ and the duty cycle of the system.
    \item How many additional cameras can be added to keep the duty cycle below 10\%?
\end{enumerate}

\section{Expected rate probability}\label{rate-probability}

We are first aiming at using the standard \textsc{Poisson}'s distribution density function to determine the expected number of people per frame, outlining therefore the expectations for each available data rate.

\smallskip

To do this, with $N$ being the number of people per frame, we will distinguish among the cases for $N = 0$, $N = 1$ and $N \geq 2$.

\medskip

Given the average rate $\lambda = 0,15$ people/frame of a \textsc{Poisson} distribution, the corresponding density function is, traditionally:

\smallskip

$P_\lambda(n) = \dfrac{\lambda^{n}}{n!} \cdot e^{-\lambda}$

\medskip

Which results in:

\smallskip

$P(r = r_0) = P(N = 0) = \dfrac{{0,15}^0}{0!} \cdot e^{-0,15} = 1 \cdot e^{-0,15} = 0,861$

\smallskip

$P(r = r_1) = P(N = 1) = \dfrac{{0,15}^{1}}{1!} \cdot e^{-0,15} = 0,15 \cdot e^{-0,15} = 0,129$

\smallskip

$P(r = r_2) = P(N \geq 2) = 1 - P(N = 0) - P(N = 1) = 0,010$

\medskip

In which we outlined the last case resulting from the remainder of the first two, as it includes all the scenarios in which the number of people per frame is $\geq 2$, also relying on standard probability properties.

\section{The complete monitoring system}

\subsubsection{Pseudocode analysis}\label{pseudocode-analysis}

The \hyperref[exercise-text]{attached pseudocode} describes a sketch of the behaviour of an overall monitoring algorithm.

\smallskip

The nodes, therefore, outline a periodic timer-based cycle, which involves the collection of a new frame from the connected \textsc{camera} capture device, which is then parsed, estimating the number of people present.

\smallskip

The payload is then formed, according to a dimensioning rule which imposes its size being $1\ kB$ (the minimum) for the case where $N = 0$ and $6\ kB$ (the maximum) for the case where $N \geq 2$.

\smallskip

Ultimately, the periodic timer restarts, analyzing a new frame every 10 seconds.

\subsubsection{Computation}

Given $R = 250\ kb/s$ and $L = 128\ Bytes$ from the text, we add the details related to the \hyperref[pseudocode-analysis]{periodic capturing and monitoring algorithm}, which underlines $d = 10\ seconds$ as the cyclic monitoring period.

\medskip

As shown during the exercise lectures, the nominal data rate ($R$) stems from:

\medskip

$R = \dfrac{L}{T_S} \Rightarrow T_S = \dfrac{L}{R} = \dfrac{128 \cdot 8\ bits}{250 \cdot 1000\ bits/second} = 4,096\ ms$

\medskip

In which we shown that the \textsc{Time Slot} duration can be easily derived by computing the inverse formula.

\medskip

In a similar way, the equivalent data rate ($r$) uses the minimum data rate as its reference, declaring, of course, subsequent ones as multiples of the reference one.

\smallskip

This translates to requiring:

\medskip

$r = \dfrac{min(l_i)\ \forall i \in [0, 1, 2]}{d} = \dfrac{1000 \cdot 8\ bit}{10\ seconds} = 800\ bits/s$

\medskip

In which $l_i$ represent the payload length (or size) in the three possible cases ($N = 0,\ 1\ or\ \geq 2$).

\medskip

This allows us to compute the inverse formula of $r$, outlining ultimately the \textsc{Beacon Interval} value:

\smallskip

$r = \dfrac{L}{BI} \Rightarrow BI = \dfrac{L}{r} = \dfrac{128 \cdot 8\ bits}{800\ bits/s} = 1,28\ seconds$

\bigskip

\label{worst-case}

When planning for the slot allocation, we cannot consider anything else than the worst-case scenario: even if, as \hyperref[rate-probability]{commented earlier}, the probability of having 3 camera nodes in the system detecting more than 1 person is very low, it still represents a theoretical possibility.

\smallskip

In such a case, we would need to allocate the space for the maximum payload size ($6\ kB$) to all the nodes.

\smallskip

Each node will therefore use $\dfrac{r_x}{r} = \dfrac{6 \cdot r}{r} = 6$ slots in the worst case.

\medskip

Moreover, the system's active state includes only the \textsc{Collision Free Part}, for which the assigned \textsc{Guaranteed Time Slots} will ultimately be:

\medskip

$N_{CFP} = \displaystyle\sum_{x = 0,\ 1,\ 2} \dfrac{r_x}{r} = 3 \cdot \dfrac{6 \cdot r}{r} = 18$ slots.

\medskip

In which we outlined the worst case when considering 3 monitoring nodes.

\bigskip

\textsc{Activity} and \textsc{Inactivity} periods will immediately follow:

\smallskip

$T_{ACTIVE} = (18 + 1) \cdot T_S = 19 \cdot T_S = 77,824\ ms$

\smallskip

In which we considered an additional slot for the \textsc{Beacon} message.

\medskip

$T_{INACTIVE} = BI - T_{ACTIVE} = 1,28\ seconds - 0,078\ seconds = 1,202\ seconds$

\medskip

From which we derive, ultimately:

\medskip

$D_{\%} = \dfrac{T_{ACTIVE}}{T_{ACTIVE} + T_{INACTIVE}} = \dfrac{T_{ACTIVE}}{BI} = 6,08\ \%$

\medskip

In which we highlighted the use of both \textsc{Active} and \textsc{Inactive} parts of the \textsc{Beacon Interval}.

\bigskip

Being the duty cycle lower than the targeted one ($10\%$), there may still be room for fitting new camera nodes.

\section{Additional camera nodes}

Let's now point our resolution towards the targeted duty cycle:

\smallskip

$D_{\%} \leq 10 \%$, which of course translates to requiring $T_{ACTIVE} \leq 128\ ms$.

\medskip

Let's now call $t$ the total number of \textsc{Time Slots} ($T_S$) that can be fit in the \textsc{Active} period, as a whole and rewrite $T_{ACTIVE}$ consequently.

\smallskip

$T_{ACTIVE} = (t + 1) \cdot T_S \leq 128\ ms$

\medskip

$t \leq \floor{\dfrac{128\ ms}{T_S} - 1} = \floor{\dfrac{128\ ms}{4,096\ ms} - 1} = 30$ slots.

\medskip

In our current implementation, $18$ slots are presently being occupied by the worst case scenario, \hyperref[worst-case]{as shown earlier}.

\smallskip

Therefore, we can use the additional $30 - 18 = 12$ \textsc{Time Slots} remaining for the additional camera nodes.

\medskip

Assuming, \hyperref[worst-case]{as before}, to adopt the worst case scenario, we can ultimately compute the allowed number of camera nodes which may fit into the monitoring system by assigning them all of the remaining \textsc{Time Slots}.

\smallskip

This translates to:

\medskip

$N_{CFP} = \floor{\displaystyle\sum \dfrac{r_x}{r} = n \cdot \dfrac{6 \cdot r}{r} = n \cdot 6} = 12$

\medskip

With $n$ being the number of allowed additional camera nodes.

\medskip

Resulting in:

\smallskip

$n = \floor{\dfrac{12}{6}} = 2$

\bigskip

The number $n$ of additional allowed camera nodes is therefore 2.
\end{document}